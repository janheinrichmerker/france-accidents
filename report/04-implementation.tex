\section{Implementation}
\label{implementation}
\begin{listing}
    \lstinputlisting[linerange={112-114,140-145}]{../src/Main.elm}
    \caption{Example of wrapping the \lstinline{update} function from \lstinline{Visualization1} inside the \lstinline{Main} module's \lstinline{update} function. Commands sent from \lstinline{Visualization1} are then mapped to the \lstinline{Main} modules types.}
    \label{listing-map-model}
\end{listing}
Our demo application is developed in the Elm programming language\footnote{\url{https://elm-lang.org/}} following the functional proramming paradigm. We implement each visualization as a sub-application in a separate Elm module~(\lstinline{Visualization1}, \lstinline{Visualization2}, \lstinline{Visualization3}) that only exposes the functions required in the Elm application architecture~(i.e., \lstinline{Model}, \lstinline{Msg}, \lstinline{init}, \lstinline{update}, \lstinline{view}) as well as a visualization name~(i.e., \lstinline{label}). The \lstinline{Main} module is then responsible for switching visualizations and any bundling messages sent from either of the sub-applications, by mapping and wrapping each visuzlization's model, messages, and view as illustrated in Listing~\ref{listing-map-model}. In addition to the modules used for displaying the application, additional modules are implemented for defining the accident model~(\lstinline{Model}), parsing accidents from the JSONL~files described in Section~\ref{data}~(\lstinline{Data}), and for utility functions and data structures~(\lstinline{Utils}, \lstinline{TimeUtils}, \lstinline{TreeUtils}, \lstinline{Partition}).

\paragraph{Model}
\begin{listing}
    \lstinputlisting[linerange={325-352}]{../src/Model.elm}
    \caption{The accident record as Elm data structure.}
    \label{listing-accident-model}
\end{listing}
We implement the accident data model by creating record types for the \lstinline[breaklines=false]{Accident}, \lstinline{Vehicle}, and \lstinline{Person} as shwon in Listing~\ref{listing-accident-model}. To type-safely model the various categorical data dimensions~(e.g., \lstinline{Intersection} or \lstinline{TravelReason}) we implement Elm set types. The Elm datastructures in the \lstinline{Model} module follow the same type definitions as the Python types from the data preprocessing code~(cf. Section~\ref{data}).

\paragraph{Data}
\begin{listing}
    \lstinputlisting[linerange={329-346}]{../src/Data.elm}
    \caption{Decoder function for parsing the person category from the JSON field.}
    \label{listing-parse-person-category}
\end{listing}
We then load the static JSONL dataset file~(cf. Section~\ref{data}) via HTTP requests and parse the response line-by-line into a list of type-safe data structures. Parsers are implemented in the \lstinline{Json.Decode} framework by creating deoders for each categorical dimension, the \lstinline{Accident} structure, the \lstinline{Vehicle} structure, and the \lstinline{Person} stucture like illustrated in Listing~\ref{listing-parse-person-category}.
Parsing the categorical dimensions requires a substantial amount of boilerplate code. For future work it might make sense to automatically generate the decoder functions given that they follow the exact same structure regardless of individual values.

\paragraph{Utilities and Additional Data Structures}
\begin{listing}
    \lstinputlisting[linerange={6-27}]{../src/Partition.elm}
    \caption{Type definitions for the \lstinline{Partition} module to partition lists based on named filters.}
    \label{listing-types-partition}
\end{listing}
\begin{listing}
    \lstinputlisting[linerange={143-155}]{../src/TimeUtils.elm}
    \caption{Helper function for stripping the year field represented in a \lstinline{Posix} timestamp.}
    \label{listing-util-remove-year}
\end{listing}
In utility modules we uncouple code that is used in multiple modules or would otherwise clutter the individual visualization modules source files.
In brief, the \lstinline{Partition} module is used to partition lists of items into labelled subsets. Multiple partitioning functions can be used to create partition trees where partitioned lists are subsequently split further until no partitioning function is left.
Using the \lstinline{Time} module, we can manipulate \lstinline{Posix} timestamps as illustrated in Listing~\ref{listing-util-remove-year}.
The \lstinline{TreeUtils} module consists of a function to convert a \lstinline{Tree} sructure into a graph tree.
And the \lstinline{Utils} module contains various functions that simplify working with \lstinline{List} values and tuples.

\paragraph{Severity Time Series}
\begin{listing}
    \lstinputlisting[linerange={161-203}]{../src/Visualization1.elm}
    \caption{Compute the y-axis value for a group of accidents.}
    \label{listing-vis1-compute-dim}
\end{listing}
\begin{listing}
    \lstinputlisting[linerange={276-284}]{../src/Visualization1.elm}
    \caption{Function to group accidents into buckets based on their \lstinline{Posix} timestamps.}
    \label{listing-vis1-buckets-timestamp}
\end{listing}
\todo{Beschreiben Sie die Implementierung ihrer Visualisierungs-Anwendung in Elm. Was war aufwändig umzusetzen, was ließ sich mit dem vorhanden Code aus den Übungen relativ einfach umsetzen? Wie sieht die Elm-Datenstruktur für das Model aus, in dem die verschiedenen Zustände der Interaktion gespeichert werden können.}

\paragraph{Geographical Map with Person Characteristics Stick Figures}
\begin{listing}
    \lstinputlisting[linerange={151-170}]{../src/Visualization2.elm}
    \caption{Mapper from the \lstinline{Accident} to its group key in preparation to grouping the accidents by grid or political region.}
    \label{listing-vis2-group-key}
\end{listing}
\begin{listing}
    \lstinputlisting[linerange={382-384,389-403,428-453,487-488}]{../src/Visualization2.elm}
    \caption{Function for drawing a single stick figure as SVG node. Dimensions~\(\gamma\) to~\(\epsilon\) omitted for readability.}
    \label{listing-vis2-stick-figure}
\end{listing}
\todo{Beschreiben Sie die Implementierung ihrer Visualisierungs-Anwendung in Elm. Was war aufwändig umzusetzen, was ließ sich mit dem vorhanden Code aus den Übungen relativ einfach umsetzen? Wie sieht die Elm-Datenstruktur für das Model aus, in dem die verschiedenen Zustände der Interaktion gespeichert werden können.}

\paragraph{Accident Type Treemap}
\begin{listing}
    \lstinputlisting[linerange={22-25,45-49}]{../src/Visualization3.elm}
    \caption{Basic model types for the accident type tree visualization.}
    \label{listing-vis3-model-types}
\end{listing}
\todo{Beschreiben Sie die Implementierung ihrer Visualisierungs-Anwendung in Elm. Was war aufwändig umzusetzen, was ließ sich mit dem vorhanden Code aus den Übungen relativ einfach umsetzen? Wie sieht die Elm-Datenstruktur für das Model aus, in dem die verschiedenen Zustände der Interaktion gespeichert werden können.}
