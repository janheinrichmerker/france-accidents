\section{Visualizations}
We propose three interactive visualizations by first analysing a set of tasks our application should solve~(cf. Section~\ref{tasks}), and then deriving specific requirements for the included visualizations~(cf. Section~\ref{requirements}). Furhtermore, we discus expressiveness, effectiveness, and appropriateness for each of the three implemented visualizations~(cf. Section~\ref{presentation}) and explain the interactivity of the visualization application~(cf. Section~\ref{interaction}).

\subsection{Analysis of Application Tasks}
\label{tasks}
\begin{table}
    \caption{Exemplary questions that should be answered by visualizions of French road accident dataset, with the typical background a person asking that question would likely have.}
    \label{table-questions}
    \begin{tabularx}{\linewidth}{cXp{0.3\linewidth}}
        \toprule
        \textbf{\#} & \textbf{Question} & \textbf{Background} \\
        \midrule
        1 & Are roads more dangerous in winter or summer? & citizen \\
        2 & Do older or younger people drive more safely? & policy maker \\
        3 & Are there hotspots of accidents in cities or rural areas? & citizen, policy maker \\
        4 & If hotspots occur, which other characteristics can explain the correlations. & policy maker \\
        5 & Do the proportion of dead and injured persons correlate? & policy maker \\
        6 & Where are unproportionately more people killed in traffic? & policy maker \\
        7 & Do dedicated bicycle lanes make roads safer for cyclists? & citizen, policy maker, infrastructure planner \\
        8 & Are wider roads safer than narrow roads? & citizen, infrastructure planner \\
        \bottomrule
    \end{tabularx}
\end{table}
Because of the diverse application background and target groups described in Section~\ref{introduction}, many tasks can be formulated to be answered by data visualizations. To narrow down the tasks, in this report, we propose a set of exemplary questions our visualizations should answer. The questions listed in Table~\ref{table-questions} also specify the most likely background a person asking that question would have. This additional context allows us to fine-tune individual visualizations by assuming background knowledge from the respective target group.
We identify 3~mental models as most important to help people answer the abovementioned questions: \Ni times and dates, \Nii geographical position, \Niii clustering or categorization.
Time and date, for example, are important to identify periodical trends, as required in question~1 in Table~\ref{table-questions}. A time series plot is able to capture the continuity of time and express a single data dimension in relation to time, which is enough for identifying the dangerousness of specific time periods.
Geographical position is crucial for understanding local patterns, as required in questions~3 or~6. While most people are very familiar with using maps as visualization for geographically positioned data, it is often not obvious how to visualize the multiple dimensions of the data points that are displayed on that map. We argue that given the space constraints of a map, the stick figures technique~\cite{PickettG1988} is appropriate because it can display a fixed number of dimensions in small space while observers can still detect patterns across the maps geographical dimensions.
And clustering accident occasions with respect to different categories is needed for comparing how different characteristics might influence road safety, as required in question~7. When working with multidimensional data, like we do with accidents, trees and treemaps~\cite{Shneiderman1992} can be a good visualization to present different, hierarchical categories.

\subsection{Visualization Requirements}
\label{requirements}
We derive 3~main goals that our visualization application should implement: \Ni to identify dangerous regions, times, and situations, \Nii to summarize trends with respect to time and location, and \Niii to offer detailed views where data is aggregated.
Identifying dangerous patterns should be supported by emphasizing road injuries or road deaths at first glance or by showing differences of accidents' characteristics with respect to visual reference points.
Accidents that follow a specific pattern should be distinguishable from other accidents.
Time-based trends should be made visible both in the long term and in the short term. Geographical trends must also include the geographical position but should not over-emphasize it to be effective.
And to avoid losing important details such as outliers, the visualizations should always feature a way to show the data with as little aggregation~(e.g., average, minimum, maximum) as possible, while still not overwhelming observers of the visualization.

\subsection{Visualization Presentation}
\label{presentation}
The visualization application proposed to solve the aforementioned tasks~(cf. Section~\ref{tasks}) consists of three interatively combined visualization pages: \Ni a time series plot to compare the number of accidents or casualties, \Nii a geographical map of icon-based stick figure visualizations, and \Niii a treemap view of hierarchically categorized accidents.

\subsubsection{Severity Time Series}
\begin{figure}
    \centering
    \includegraphics[width=0.9\linewidth]{figures/time-series-2-to-1-killed-or-injured-absolute-never-per-year.png}
    \caption{Time series of the number of killed or injured persons per year from~2005 to~2020 with a 2:1~aspect ratio.}
    \label{figure-time-series-killed-injured-per-year}
\end{figure}
\begin{figure}
    \centering
    \includegraphics[width=0.9\linewidth]{figures/time-series-2-to-1-killed-or-injured-absolute-by-year-per-month.png}
    \caption{Time series of the number of killed or injured persons per month with a 2:1~aspect ratio, by aggregating across all years~2005--2010.}
    \label{figure-time-series-killed-injured-by-year-per-month}
\end{figure}
\begin{figure}
    \centering
    \includegraphics[width=0.9\linewidth]{figures/time-series-banking-45-unharmed-relative-never-per-quarter.png}
    \caption{Time series of the proportion of unharmed persons of all persons involved in accidents per quarter with banking to~45\(^\circ\).}
    \label{figure-time-series-banking-unharmed-relative-per-quarter}
\end{figure}
The first visualization we introduce is a time series plot of the prevalence and severity of accidents across time.
Figures~\ref{figure-time-series-killed-injured-per-year}, \ref{figure-time-series-killed-injured-by-year-per-month}, and~\ref{figure-time-series-banking-unharmed-relative-per-quarter} illustrate this visualization.
\todo{Präsentieren sie die visuellen Abbildungen und Kodierungen der Daten und Interaktionsmöglichkeiten.
Sie müssen begründen, warum und wie gut ihre Designentscheidungen die erstellten Anforderungen erfüllen.
Weiterhin müssen sie begründen, warum die gewählte visuelle Kodierung der Daten für das zu lösenden Problem passend ist.
Typische Argumente würden hier auf Wahrnehmungsprinzipien und Theorie über Informationsvisualisierung verweisen. 
Die besten Begründungen diskutieren explizit die konkrete Auswahl der Visualisierungen im Kontext von mehreren verschiedenen Alternativen. 
Machen Sie hier nicht den Fehler, einfach nur Visualisierung aus den vorgegebenen Bereichen zu diskutieren, weil das in der Regel nicht sinnvoll ist.
Wenn sie sich für einen Scatterplot entschieden haben, ist ein Zeitreihendiagramm in der Regel keine Alternative.
Diskutieren Sie also nicht einfach Zeitreihendiagramme, weil sie in den Anforderungen an das Projekt neben Scatterplots stehen, sondern suchen sie nach echten alternativen Visualisierungen, die zum Aufbau eines vergleichbaren mentalen Modells führen.
Diskutieren Sie die Expressivität und die Effektivität der einzelnen Visualisierungen.}

\subsubsection{Geographical Map with Person Characteristics Stick Figures}
\begin{figure}
    \centering
    \includegraphics[width=0.6\linewidth]{figures/stick-figures-by-grid-20-20-average.png}
    \caption{Stick figure icons of average person characteristics on a map of France after partitioning the map into a \(20 \times 20\)~grid.}
    \label{figure-stick-figures-grid-average}
\end{figure}
\begin{figure}
    \centering
    \includegraphics[width=0.6\linewidth]{figures/stick-figures-by-grid-10-10-xray.png}
    \caption{Stick figure icons of person characteristics on a map of France after partitioning the map into a \(10 \times 10\)~grid. Each group's stick figures are overlayed in x-ray style.}
    \label{figure-stick-figures-grid-xray}
\end{figure}
\begin{figure}
    \centering
    \includegraphics[width=0.6\linewidth]{figures/stick-figures-by-department-average.png}
    \caption{Stick figure icons of average person characteristics on a map of France after partitioning the map into France's official Departements.}
    \label{figure-stick-figures-departments-average}
\end{figure}
To visualize geographical characteristics of involved persons, we display stick figure icons~\cite{PickettG1988} of aggregated person groups on an equirectanguular geographical map projection of France\footnote{\url{https://commons.wikimedia.org/wiki/File:France_location_map-Regions_and_departements-2016.svg}}.
Users can either choose to display one single stick figure per group, visualizing the average values of that group, or to display each instance within a group as a separate stick figure, following the x-ray design pattern.
\todo{which dimensions}
Figures~\ref{figure-stick-figures-grid-average}, \ref{figure-stick-figures-grid-xray}, and~\ref{figure-stick-figures-departments-average} illustrate this visualization.
\todo{Präsentieren sie die visuellen Abbildungen und Kodierungen der Daten und Interaktionsmöglichkeiten.
Sie müssen begründen, warum und wie gut ihre Designentscheidungen die erstellten Anforderungen erfüllen.
Weiterhin müssen sie begründen, warum die gewählte visuelle Kodierung der Daten für das zu lösenden Problem passend ist.
Typische Argumente würden hier auf Wahrnehmungsprinzipien und Theorie über Informationsvisualisierung verweisen. 
Die besten Begründungen diskutieren explizit die konkrete Auswahl der Visualisierungen im Kontext von mehreren verschiedenen Alternativen. 
Machen Sie hier nicht den Fehler, einfach nur Visualisierung aus den vorgegebenen Bereichen zu diskutieren, weil das in der Regel nicht sinnvoll ist.
Wenn sie sich für einen Scatterplot entschieden haben, ist ein Zeitreihendiagramm in der Regel keine Alternative.
Diskutieren Sie also nicht einfach Zeitreihendiagramme, weil sie in den Anforderungen an das Projekt neben Scatterplots stehen, sondern suchen sie nach echten alternativen Visualisierungen, die zum Aufbau eines vergleichbaren mentalen Modells führen.
Diskutieren Sie die Expressivität und die Effektivität der einzelnen Visualisierungen.}

\subsubsection{Accident Type Treemap}
\begin{figure}
    \centering
    \includegraphics[width=0.6\linewidth]{figures/tree-treemap-weather-light-condition.png}
    \caption{Tree map of accidents by weather and light condition.}
    \label{figure-treemap-weather-light-condition}
\end{figure}
\begin{figure}
    \centering
    \includegraphics[width=0.6\linewidth]{figures/tree-treemap-location-type-traffic-regime-intersection-type-road-curvature-road-profile-dedicated-line-road-category.png}
    \caption{Tree map of accidents by location type, traffic regime, intersection type, road curvature, road profile, dedicated lines, and road category.}
    \label{figure-treemap-many}
\end{figure}
Figures~\ref{figure-treemap-weather-light-condition} and~\ref{figure-treemap-many} illustrate this visualization.
\todo{Präsentieren sie die visuellen Abbildungen und Kodierungen der Daten und Interaktionsmöglichkeiten.
Sie müssen begründen, warum und wie gut ihre Designentscheidungen die erstellten Anforderungen erfüllen.
Weiterhin müssen sie begründen, warum die gewählte visuelle Kodierung der Daten für das zu lösenden Problem passend ist.
Typische Argumente würden hier auf Wahrnehmungsprinzipien und Theorie über Informationsvisualisierung verweisen. 
Die besten Begründungen diskutieren explizit die konkrete Auswahl der Visualisierungen im Kontext von mehreren verschiedenen Alternativen. 
Machen Sie hier nicht den Fehler, einfach nur Visualisierung aus den vorgegebenen Bereichen zu diskutieren, weil das in der Regel nicht sinnvoll ist.
Wenn sie sich für einen Scatterplot entschieden haben, ist ein Zeitreihendiagramm in der Regel keine Alternative.
Diskutieren Sie also nicht einfach Zeitreihendiagramme, weil sie in den Anforderungen an das Projekt neben Scatterplots stehen, sondern suchen sie nach echten alternativen Visualisierungen, die zum Aufbau eines vergleichbaren mentalen Modells führen.
Diskutieren Sie die Expressivität und die Effektivität der einzelnen Visualisierungen.}

\subsection{Interaction}
\label{interaction}
One of the requirements we formulated~(cf. Section~\ref{requirements}) is to be able to explore details. 
We see an interaction between the three proposed visualizations as an oportunity to fulfil that goal and to encourage viewers to explore differences for groups of similar accidents~(both in terms of geographical region as well as for characteristics).
Therefore viewers are given the option to select filtered subsets of the accidents to explore them in a different visualization.
This filtering is possible in two places: \Ni by selecting geographical clusters from the stick figure visualization map and \Nii by selecting nodes from the tree visualized in the accident characteristics treemap.
The first case is especially useful for citizens who want to focus their exploration on just their home town~(shown in Figure~\todo{example screenshot}). They can then, for example, switch to the severity time series to find out when to avoid driving specifically in their home town or find out what the most common accident types are by switching to the tree visualization. The second case is most useful for policy makers or public infrastructure planners because they can first identify a target problem they want to mitigate in the treemap visualization. Then they could, for example, explore where that type of accident occurs most often~(e.g., to plan an infrastructure project, as exemplified in Figure~\todo{example screenshot}) or what persons are often involved~(e.g., to educate or regulate that group of persons). Filtering by local characteristics such as road curvature or slopes can also be useful for identifying periodical patterns with the time series visualization, for example to apply measures only in summer or winter.
As the three visualizations are already complex and feature high-dimensional data, we refrain from embedding individual visualizations into others as pop-up.
We also planned to implement a time range selector to filter accidents by time or season, but had to omit that filter due to the time constraints.
