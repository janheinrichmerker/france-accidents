\section{Conclusion and Future Work}
The three visualizations proposed in this report represent a valuable example on how to interactively visualize datasets in the Elm framework.
Three use-cases based on analyzing the application's target groups illustrate our visualization application's expressiveness and effectiveness.
It can \Ni help citizens to avoid commuting in more dangerous months or places, \Nii support government in targeted education or regulation of road safety, and \Niii inspire infrastructure planners in identifying problematic or needed infrastructure.
Particularly striking are the trend that men are more often involved in accidents and that against the intuition driving in winter can be considered safer than in summer.
Furthermore, we make the French accident dataset more accessible by preprocessing it into a human-~and machine-readable format.
By publishing our source code\footnote{\url{https://github.com/janheinrichmerker/france-accidents/}} and an online demo\footnote{\url{https://janheinrichmerker.github.io/france-accidents/}}, we encourage further research of accident characteristics.
Interesting directions of future work include exploring the recursive design pattern for the time series, improving the map design, exploring other icon-based approaches, extending our framework with more dimensions and filters~(we use just some of the many fields from the available data model), optimizing our application's source code to be able to work with larger datasets, and expanding the findings to European or international roads---to finally make roads safe for everyone.
