\section{Related Work}
Geographical plots have already been used in the past to visualize multi-dimensional data of road accidents.
\textcite{LeLL2020} visualize road traffic accidents in the city of Hanoi, Vietnam from~2015 to~2017, based on data from the Transport Police Department of Hanoi.
For their visualizations they split the 1132~accidents from the small dataset per season and then assign each accident a severity index that is calculated as a weighted sum of the number of light, severe, and fatal accidents~\cite{GeurtsWBV2004}.
For each season and time of day \citeauthor{LeLL2020} then transform accidents into a grid using Kernel Density Estimation~\cite{Anderson2009}. Accidents in this grid are then displayed on a street map of Hanoi where they plot the severity index value per grid cell~(Figure~3 of their paper~\cite{LeLL2020}).
Similar to our work, \citeauthor{LeLL2020} also group the geographically tagged accident records into grid cells of equal size. However while their grid size is relatively small and dynamically determined~(using a bandwidth of 1\,km for Kernel Density Estimation), our grid is relatively large~(approximately 50\,km). Also they only encode a single dimension per plotted data point by the point's size while in our visualization multiple dimensions are plotted in stick figure style~\cite{PickettG1988}.
Moreover the map by \citeauthor{LeLL2020} features information about infrastructure by including street paths and visualizing different types of roads. We did not find a freely available street map of France, and hence only use a simple map that only indicates different Departements inside France. It would be an interesting direction of future research to integrate a map of the French public road network into our visualizations.

\textcite{LavravcJTK2008} also visualize traffic accidents and their seasonality by analyzing a database of 453\,451~accident that occured in Slovakia from~1995 to~2005.
To indicate seasonal influences on the number of accidents, they propose a two-dimensional heatmap~(Figure~2 of their paper~\cite{LavravcJTK2008}) by months of the year and weekdays. By using a greyscale color coding for the number of accidents they circiumvent issues with color-blindness. However it is even more difficult to visually map from the different shades of grey to the number of accidents by comparing with the plot's legend. Because the shades look so similar, for example, it is nearly impossible to distinguish an accident count of~6001 from~6501.
Neither the visualization by \citeauthor{LavravcJTK2008} nor the visualization by \Citeauthor{LeLL2020} feature user exploration by interactive filtering and changing of visualization styles.
